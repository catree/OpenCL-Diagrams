%!TEX root = ../main.tex

\begin{figure}[p]
\centering
\begin{tikzpicture}[scale=1.0,transform shape]

  % Parameters
  \def\wtab{21.8em}
  \def\htab{7.25em}
  \def\celldim{0.7}

  % Styles
  \tikzstyle{table}=[draw,thick,rounded corners,text centered,minimum width=\wtab,minimum height=\htab,inner sep=0.3cm]
  \tikzstyle{cell}=[anchor=north west,text centered,minimum width=1.8em-0.05em,minimum height=1.8em-0.05em,inner sep=0.3cm]
  \tikzstyle{shade}=[draw,thick,anchor=north west,fill=gray,opacity=0.3,pattern=north west lines,text centered]
  \tikzstyle{sep}=[dashed,line width=0.7pt]
  \tikzstyle{to}=[->,>=stealth',shorten <=1pt,shorten >=1pt,line width=2.5pt,red,cap=round]

  % \frame{<id>}{<pos>}[<fill-color>][<rot-angle>]
  \DeclareDocumentCommand \frame { m m o O{0} }{%
    \path #2 node[table,rotate=#4,anchor=north west,fill=\IfValueTF{#3}{#3}{none}] (frame-#1) {};
    \foreach \x in {1,2,...,11}
      \draw[sep,rotate=#4] ($(frame-#1.north west)+(\x*\celldim,0)$) -- ++(0,-\htab);
    \foreach \y in {1,2,3}
      \draw[sep,rotate=#4] ($(frame-#1.north west)+(0,-\y*\celldim)$) -- ++(\wtab,0);
  }

  % \cell{<pos>}{<cell-x-id>}{<cell-y-id>}{<color>}[<rot-angle>][<north-west-arc>][<south-west-arc>][<north-east-arc>][<south-east-arc>]
  \DeclareDocumentCommand \cell { m m m m O{0} O{0pt} O{0pt} O{0pt} O{0pt} }{%
    \path[rotate=#5] ($#1+({#2*\celldim},{-#3*\celldim})$) node[cell,append after command={\pgfextra \draw[draw=none,sharp corners,fill=#4] (\tikzlastnode.north) [rounded corners=#8] -| (\tikzlastnode.east) [rounded corners=#9] |- (\tikzlastnode.south) [rounded corners=#7] -| (\tikzlastnode.west) [rounded corners=#6] |- (\tikzlastnode.north); \endpgfextra}] {};
  }

  % \table{<id>}{<pos>}[<rot-angle>]
  \DeclareDocumentCommand \table { m m O{0} }{%
    % R channel
    \foreach \x in {1,2,...,10}
      \cell{#2}{\x}{0}{Red!90}[#3];
    \cell{#2}{0}{0}{Red!90}[#3][4.5pt]
    \cell{#2}{11}{0}{Red!90}[#3][0pt][0pt][4.5pt]
    % G channel
    \foreach \x in {0,1,...,11}
      \cell{#2}{\x}{1}{LimeGreen!80}[#3];
    % B channel
    \foreach \x in {0,1,...,11}
      \cell{#2}{\x}{2}{NavyBlue!80}[#3];
    % A channel
    \foreach \x in {1,2,...,10}
      \cell{#2}{\x}{3}{Yellow!35}[#3];
    \cell{#2}{0}{3}{Yellow!35}[#3][0pt][4.5pt]
    \cell{#2}{11}{3}{Yellow!35}[#3][0pt][0pt][0pt][4.5pt]
    % Frame
    \frame{#1}{#2}[none][#3]
  }

  % \shade{<pos>}{<x-cells>}{<y-cells>}{<label>}
  \DeclareDocumentCommand \shade { m m m m }{%
    \path #1 node[shade,minimum width=#2*1.815em,minimum height=#3*1.8em] (shade) {};
    \node[text=Fuchsia] at (shade) {\sffamily\bfseries\Large #4};
  }

  % ===========================================================================
 
  % Draw tables
  \table{0}{(0,0)}[90]
  \table{1}{($(frame-0.south)+(0.8*\htab,0.5*\htab)$)}[0]

  % Draw transition arrow
  \draw[to] ($(frame-0.south)+(0.1*\htab,0)$) -- node[midway,above] {\sffamily\LARGE\bfseries T} ($(frame-1.west)+(-0.1*\htab,0)$);

  % Draw shades
  \shade{($(frame-0.north east)+(0,-\celldim-0.01)$)}{4}{1}{pixel}
  \shade{($(frame-1.north west)+(0,-\celldim+0.01)$)}{12}{1}{channel}

\end{tikzpicture}
\caption{Representation of an image transpose.}
\label{fig:rgba-transpose}
\end{figure}
